\section{SPI Système}
\subsection{Description}

Hamster peut communiquer en tant qu'esclave SPI.

L'interface SPI utilisée est configurée en mode 0 (CPHA=0, CPOL=0). Les caractèristiques de cette interface SPI sont décrites \cref{tab-spi-parameters}.
Le système capture donc les données sur front \emph{montant} de l'horloge SPI et envoie de nouvelles données sur front \emph{descendant}.
Il est nécessaire que l'horloge SPI \sclk soit au moins 5\todo{check} fois plus lente que l'horloge système \clksys.

D'une manière générale, on considèrera comme fréquence maximale pour le SPI $\freq{\sclk}$ = 10 \mega\hertz

Un registre de test est disponible à l'adresse 0xA9 et vaut toujours 0xCAFE.

\begin{table}[htbp]
    \centering
\begin{spectable}
    Fréquence d'horloge & \freq\sclk & & & 10 \mega\hertz \\
    Phase d'horloge& CPHA & & 0 & \\
    Polarité d'horloge & CPOL & & 0 & \\
    \hline
    Bits par mot & & & 16 & \\
    Premier bit envoyé & & & MSB & \\ 
    \hline
\end{spectable}
\caption{Paramètres du SPI système}
\label{tab-spi-parameters}
\end{table}


\subsection{Commandes}

\subsubsection{0x01 - READ}
La commande SPI READ permet de lire une ou plusieurs valeurs consécutives dans la banque de registre de Hamster.
Le format général d'une lecture simple est décrit \cref{fig-wave-spi-read-comptest}. 
Le maître initie la commande en passant \csn à 0 et présente le premier bit (MSB) de la commmande READ \emph{avant} le front montant de l'horloge.


\begin{figure}[h]
    \centering
    \def\svgwidth{17cm}
    \import{pdftex/}{wave_spi_read_test.pdf_tex}
    \caption{Exemple de la lecture du registre de test}
    \label{fig-wave-spi-read-comptest}
\end{figure}