\subsection{Definitions relatives aux commandes des phases}

Les moteurs cible de Hamster sont les moteurs triphasés, avec un branchement en étoile comme représenté \cref{fig-schema-phases-moteur}.
On peut considérer la présence de trois bobines, branchées en un point commun, nommé point milieu (cercle vert).
Chaque extrémité libre de ces bobines (carrés jaunes) peuvent être :
\begin{itemize}
    \item reliée à une source de courant (carré encadré en rouge)
    \item reliée à une masse (carré encadré en bleu)
    \item laissé flotante (carré encadré en gris, et bobine en pointillés)
\end{itemize}

On considèrera le flux magnétique généré par le passage du courant en direction du point central comme étant négatif (pôle N, flèche bleue).
De même, le flux magnétique provoqué par le passage du courant du point milieu vers l'extrémité libre de la bobine (flèche rouge) sera considéré positif (pôle P).

\begin{figure}[h]
    \centering
    \import{pdftex/}{triphase_definition.pdf_tex}
    \caption{Schéma simplificatif des phases d'un moteur}
    \label{fig-schema_phases_moteur}
\end{figure}

Dans la mesure où le point milieu n'est généralmement pas accessible et pour réduire la complexité de l'électronique, seules les extrémités libres de la bobines seront exposées a l'exterieur du moteur.
Ainsi, seules ces extrémités pourront être reliées à un système de pilotage.

\paragraph{}
On nommera \emph{phases} les bobines du moteur. Dans la mesure où un moteur peut avoir plusieurs bobines par phase, 

\paragraph{}
On nommera \emph{commande d'une phase} l'état (flottante, connectée  à une source ou connectée à la masse) d'une phase et, par extension, l'action de placer une phase dans l'un de ces états.

\paragraph{}
On notera \phase{} l'ensemble des phases. On notera \phase{x} une phase particulière où $x$ est le nom de la phase. On considèrera avoir trois phases nommées A, B et C et donc
$$\phase{} = \{\phase{A} , \phase{B}, \phase{C}\} $$
Toute autre lettre placée en indice (par exemple \phase{x}) indique une phase quelconque. Toute autre signification sera indiquée explicitement.

\paragraph{}
On représentera les différents états de phase à l'aide des symboles suivants :
\begin{itemize}
    \item Phase alimentée par une source de courant : \phhi
    \item Phase liée à la masse : \phlo
    \item Phase flottante : \phoff
\end{itemize}
On notera l'association d'une phase à un état en utilisant une notation en exposant.
Par exemple : \phstate{x}{\phhi} indique que la phase $x$ est alimentée par une source de courant.
Tout autre symbole (par exemple \phstate{x}{\alpha}) indique, sauf mention contraire, un état quelconque.

On nommera \emph{commande du moteur} un vecteur \cmdmot défini par
$$ \cmdmot = \left(\phstate{A}{\alpha},\phstate{B}{\beta},\phstate{C}{\gamma}\right)$$

