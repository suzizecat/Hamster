\section{Commande moteur}
Si l'on considère le chronogramme \cref{fig:chrono_centered_motor}, il est possible de définir les délai suivants :
\begin{itemize}
    \item \tstep : Durée passée sur une étape de pilotage de moteur.
    \item \tpre : Durée entre la sélection des phases à alimenter et le début de l'envoi de puissance dans le moteur.
    \item \ton  : Durée active du pilotage du moteur (puissance est transmise)
    \item \tpost: Durée entre la fin de la periode active du pilotage et le changement de sélection de phases.
    \item \toff : Durée entre deux activation de la puissance
\end{itemize}

\begin{figure}[h]
    \def\svgwidth{17cm}
    \import{pdftex/}{wave_pwr_cmd.pdf_tex}
    \caption{Chronogramme}
    \label{fig:chrono_centered_motor}
\end{figure}

Etant donné \cref{fig:chrono_centered_motor}, et afin d'assurer un bon fonctionnement du système, il est possible de poser les contraintes suivantes :
$$\toff \ge t_{\mathit{SW_{\mathit{MOS}}}}$$

Pour tenter de centrer la commande, on pourra utiliser :
$$\tpre^{+1} = \frac{\tpre + \tpost + \Delta \ton}{2}$$
Cependant, on pourra tenter l'implémentation sans le facteur $\Delta\ton$ car il peut être complexe d'estimer ce coefficient.
Comme l'utilisation de ce facteur risque de sur-estimer la commande, il sera intéressant de pouvoir désactiver dynamiquement l'utilisation de ce facteur.
